%%%%%%%% ICML 2021 LATEX SUBMISSION FILE %%%%%%%%%%%%%%%%%

\documentclass{article}

% Recommended, but optional, packages for figures and better typesetting:
\usepackage{microtype}
\usepackage{graphicx}
\usepackage{subfigure}
\usepackage{booktabs} % for professional tables

% hyperref makes hyperlinks in the resulting PDF.
% If your build breaks (sometimes temporarily if a hyperlink spans a page)
% please comment out the following usepackage line and replace
% \usepackage{icml2021} with \usepackage[nohyperref]{icml2021} above.
\usepackage{hyperref}

% Attempt to make hyperref and algorithmic work together better:
\newcommand{\theHalgorithm}{\arabic{algorithm}}

% Use the following line for the initial blind version submitted for review:
\usepackage{icml2021}

% If accepted, instead use the following line for the camera-ready submission:
%\usepackage[accepted]{icml2021}

% The \icmltitle you define below is probably too long as a header.
% Therefore, a short form for the running title is supplied here:
\icmltitlerunning{Conversational Speech Transcription
Using Context-Dependent Deep Neural Networks}

\begin{document}

\twocolumn[
\icmltitle{Conversational Speech Transcription
Using Context-Dependent Deep Neural Networks}


\icmlsetsymbol{equal}{*}

\begin{icmlauthorlist}
\icmlauthor{Aeiau Zzzz}{equal,to}
\end{icmlauthorlist}


\icmlcorrespondingauthor{Cieua Vvvvv}{c.vvvvv@googol.com}
\icmlcorrespondingauthor{Eee Pppp}{ep@eden.co.uk}

% You may provide any keywords that you
% find helpful for describing your paper; these are used to populate
% the "keywords" metadata in the PDF but will not be shown in the document
\icmlkeywords{Machine Learning, ICML, Deep Neural}

\vskip 0.3in
]


\begin{abstract}
Context-Dependent Deep-Neural-Network HMMs, or CD-DNN-HMMs, combine the
classic artificial-neural-network HMMs with traditional context-dependent acoustic modeling and deep-belief-network pre-training.
CD-DNN-HMMs greatly outperform conventional CD-GMM (Gaussian mixture model)
HMMs: The word error rate is reduced by up
to one third on the difficult benchmarking
task of speaker-independent single-pass
transcription of telephone conversations.
\end{abstract}

\section{Introduction }
\label{submission}
Context-dependent deep-neural-network HMMs (CD-DNN-HMMs) are a recently proposed acoustic-modeling technique for HMM-based speech recognition [1, 2]
that combines three techniques: the hybrid approach
of modeling HMM state emission densities through
scaled likelihoods from an MLP [3]; traditional acoustic co-articulation modeling of speech through context-dependent phoneme models (crossword triphones with
tied states); and deep networks, leveraging Hinton’s
deep-belief-network (DBN) pre-training procedure.
The power of this model was first shown through a 16%
relative recognition error reduction over conventional
CD-GMM-HMMs on a business search task [1, 2]. This
work describes our subsequent efforts [4] on scaling it
up in terms of training-data size (from 24 hours to
309), model complexity (from 761 output classes to
9304), depth (up to 9 hidden layers), and task (from voice queries to speech-to-text transcription). The
model achieves a one-third word-error reduction on the
publicly available benchmark of phone-call transcription (Switchboard 2000 NIST Hub5/RT03S-FSH).


\section{The Context-Dependent Deep
Neural Network HMM}
In HMM-based large-vocabulary speech recognition,
speech is modeled by hidden Markov models (HMMs),
where each word’s HMM is decomposed into phoneme
HMMs. These are commonly three-state left-to-right
HMMs, where each state’s emission probability is a
mixture of Gaussians (GMM). Co-articulation is modeled by context-dependent (CD) phonemes, such as
triphones. Due to data scarcity, triphone states are
commonly tied with similar other states.
A limitation of GMMs is their difficulty to use high-dimensional features, such as multiple consecutive
frames of short-term spectral features. To address this,
it was proposed in the early 90’s to replace GMMs
with artificial neural networks (ANNs). The ANNs are
trained to classify observation vectors into HMM state
labels [3], and state posteriors are converted to scaled
likelihoods for use as HMM state emissions. However,
these early attempts were limited to shallow models
(1–2 hidden layers) and monophone states as ANN
outputs (even when CD phones were modeled) [5, 6].
The CD-DNN-HMM extends these hybrid ANN-HMMs two-fold: First, we model tied triphone states
directly. It was long assumed that thousands of triphone states were too many to be accurately modeled
by an MLP, but [1] has shown that it works very well.
Secondly, we use a deep MLP with with many hidden layers. Many layers of simple non-linearities can 
model complicated non-linearities and are more efficient in representing structures since lower-layer feature detectors can be reused by the higher-layer feature
detectors. Also, each layer is constrained by the adjacent layers and so it is less likely to cause over-fitting
(although it is more likely to cause under-fitting).
The key enablers to the training of these were the deep
belief network (DBN) pre-training algorithm proposed
by Hinton [7], as well as the advent of affordable, massively parallel computing devices (GPGPUs). Algorithm 1 summarizes the training procedure [4]. First,
a conventional CD-GMM-HMMs is trained. Secondly,
the DNN, after initialization as a DBN, is trained as a
frame classifier, where the class labels are state labels
assigned to each input frame through forced alignment
using the CD-GMM-HMM. Midway, the alignment is
updated once using the DNN model.

\onecolumn

\begin{tabular}{ |c|c|c|c|c|c|c } 
 \hline
 acoustic model and training & recognition mode & RT03S & Hub5’00 & voice mails & teleconf \\ 
 \hline
  \hline
CD-GMM 40-mix, SWB 309h & single-pass SI & FSH=27.4 SW=37.6 & 23.6 & 30.8 & 33.9 \\ 
 \hline
  \hline
 CD-DNN 7 layers x 2048, SWB 309h (this work)\\ single-pass SI 18.5 27.5 16.1 22.9 24.4\\
(rel. change CD-GMM → CD-DNN) & single-pass SI & FSH=18.5 SW=27.5 & 16.1 & 22.9 & 24.4 \\ 
 \hline
  \hline
 CD-GMM 72-mix, Fisher 2000h & multi-pass adaptive &  FSH=18.6 SW=25.2 & 17.1 & - & - \\ 
 \hline
\end{tabular}

\emph{Table 1. Standard CD-GMM-HMM vs. CD-DNN-HMM for single-pass speaker-independent recognition on five speech-to text test sets (word-error rates in $\%$), and for comparison our group’s best-ever CD-GMM-HMM result for three set}.
\twocolumn
\\
\section{Experimental Results}
We evaluate the effectiveness of CD-DNN-HMMs on
speech-to-text transcription of telephone conversations, a considerably difficult task. We use the publicly available 309-hour ‘SWBD-I’ training set and associated benchmark sets, as well as two in-house sets.
Recognition is single-pass without speaker adaptation.
Table 1 shows that compared to our discriminatively
trained CD-GMM-HMM baseline, the word-error rate
(WER) on the ‘RT03S-FSH’ benchmark drops from 27.4 
$\%$
to 18.5$\%$—a rather significant one-third reduction. Much of the gain carries over to less well-matched
sets (voicemail, teleconferences). The 309h CD-DNN-HMM system also reaches our best multi-pass system (18.6$\%$, last row), which uses 6 times as much acoustic
training data and speaker adaptation.
Further experiments show that the deep network is indeed critical—a shallow 1-hidden-layer network using
the same number of parameters as the 7-hidden-layer
one leads to five percentage points worse word-error
rate. We also find that as an alternative to DBN pretraining, it is possible to discriminatively pre-train the
model in a supervised layer-growing fashion


\section{Conclusion}
By using CD-DNN-HMMs, a one-third word-error reduction has been achieved on a difficult benchmark
task, compared to a discriminatively trained conventional CD-GMM-HMM [4]. Recent improvements on
smaller tasks [1, 2] do carry over to larger corpora and
speech-to-text transcription. The remarkable accuracy
gains are due to three factors: direct modeling of tied
triphone states through the DNN; effective exploitation of neighbor frames by the DNN; and the efficient
and effective modeling ability of deeper networks.

\section*{References}

[1] D. Yu et al., “Roles of Pretraining and Fine-Tuning
in Context-Dependent DNN-HMMs for Real-World
Speech Recognition,” Proc. NIPS Workshop on Deep
Learning and Unsupervised Feature Learning, 2010.
\\ 

[2] G. Dahl et al., “Context-Dependent Pre-Trained Deep
Neural Networks for Large Vocabulary Speech Recognition”, IEEE Trans. Speech and Audio Proc., Special
Issue on Deep Learning for Speech Processing
 \\ \\ \\ \\ \\ \\ \\ \\ \\

[3] S. Renals et al., “Connectionist Probability Estimators in HMM Speech Recognition,” IEEE
Trans. Speech and Audio Proc., January 1994.
\\ 

[4] F. Seide, G. Li, and D. Yu, “Conversational Speech
Transcription Using Context-Dependent Deep Neural
Networks,” Interspeech, 2011.
\\

[5] H. Franco et al., “Context-Dependent Connectionist Probabilty Estimatation in a Hybrid Hidden
Markov Model–Neural Net Speech Recognition System,” Computer Speech and Language, vol. 8, 1994.
\\

[6] J. Fritsch et al., “ACID/HNN: Clustering Hierarchies
of Neural Networks for Context-Dependent Connectionist Acoustic Modeling,” Proc. ICASSP, May 1998.
\\

[7] G. Hinton et al, “A Fast Learning Algorithm for Deep
Belief Nets”, Neural Computation, vol. 18, 2006.
\\

[8] D. Rumelhart et al., “Learning Representations By
Back-Propagating Errors,” Nature, vol. 323, 1986

\bibliographystyle{icml2021}


\end{document}

